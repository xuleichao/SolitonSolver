\documentclass{article} 
\usepackage[UTF8, scheme = plain]{ctex}
\usepackage{multicol}
\usepackage{fontspec}
\usepackage{cite}

\begin{document}

\title{500篇孤子摘要总结}

\author{Leichao Xu and Tie Zhang}

\date{\today}



\maketitle
%摘要
\begin{abstract}
从web of knowledge中搜索关键词soliton, 导出排名500的文献及摘要,通过阅读部分文献,了解当前孤子研究的主流方向,
例如:特殊材料中孤子的产生以及其稳定性研究,光纤中孤子解的研究,利用孤子来产生逻辑门的研究,以及求孤子解的问题等,另外有一些前沿性的研究,如
超光速孤子的概念,量子图概念。此外,还有一些研究宏观世界下孤子的问题,例如高速运行的火车头部的孤子解,海洋中鱼群相关的孤子理论以及黑洞孤子概念等。
\end{abstract}
\begin{multicols}{2}
\section{Indroduction}
孤子最开始来源于水波波峰长时间保持形状传播,之后产生了非线性的研究,即孤子解属于非线性系统的解。对于微观系统,如玻色爱因斯坦系统(BEC),非线性周期系统,会产生孤子解,同时相同带隙中的孤子解与对应非线性布洛赫波有对应关系\cite{Zhang.2009}。而对于宏观系统,孤子解依然可以解释一些自然现象,例如对黑洞中孤子的研究\cite{dymnikova2020regular},海洋相关的孤子研究\cite{stepanyants2020nonlinear,zhao2019the},海洋中鱼群与孤子的关系\cite{soliton_russion},高速运行的火车头部的空气研究\cite{soliton_russion}。无论在微观领域还是宏观领域,孤子的研究与应用非常广泛,本篇综述将从孤子的研究现状,研究应用,以及当前孤子研究的新方向与新领域做大致的阐述,一方面作为文献摘要阅读的总结,也为孤子研究的方向确定做总结,该综述的严谨性会比较差,有错误的对方或者有争议的地方,望指出或一起讨论。

%孤子研究现状
\section{孤子研究现状}
孤子研究大致可以分为两个方向,第一是理论研究与数值模拟,第二是实验研究。数值模拟围绕求解非线性薛定谔方程来得到孤子解,然后对孤子的稳定性展开叙述。对孤子的求解不仅限于亮孤子,而且还有暗孤子\cite{article,2017Decay},暗孤子只孤子解为凹陷型,孤子的形状也是一个研究方向,主要研究孤子解的曲线弧度\cite{Umbetova_2019}。对于孤子的求解,本质上是求解非线性薛定谔方程\cite{soliton_russion2},有人也致力于求解精确的解析解\cite{2020Exact,inproceedings},这类问题偏向于数学方向,对于物理研究,数值解应该是首选,毕竟物理是要解释现象。得到孤子解后就需要讨论孤子解的稳定性,不稳定孤子在实验中难以观测,在一定时间内可以保持固定形状的孤子被称为稳定孤子,而不稳定的孤子也有其特性,例如有研究孤子超光速但是具有不稳定性\cite{1937-1632_2020_8_2285},孤子的稳定性研究同样有数值模拟微扰法\cite{article_stable}。数值方法有分步傅里叶方法\cite{article_step},通过该方法可以直观的看到孤子随时间的演化。
\section{实验研究}
理论研究与实验研究不能分割,实验结论需要用理论去解释,从而去促进理论发展、去发现更多有趣有用的实验现象,相反,理论需要用实验去验证,做理论物理不能脱离实际,如果理论中需要的外界环境在现实中根本不存在或者难以制备,那么理论的研究意义将大打折扣。这里的实验研究指将理论应用到实际或者研究理论落地。
\subsection{光孤子实验研究}
孤子的特性之一是稳定传播,因为可以作为信息传播的载体,故而光学孤子\cite{mariana2020latest,kulakovskii2019coherence}和光纤中孤子\cite{article_step,morin2020sub,chen2020toward,mayteevarunyoo2020spatiotemporal}的研究是热点。光学研究中的参数对理论研究意义重大,理论计算中的某些常数需要从实验中获取,否则可能会脱离实际,例如Zhaoyiwu讨论了关于一种单晶体\cite{zhao2020generation}论文的参数可能会影响到带宽从而影响到孤子解的存在与否,孤子持续时长直接影响到孤子的稳定性。

既然光孤子可以稳定的存在,人们自然可以想去操纵光孤子,例如操作光孤子实现逻辑门\cite{jandieri2020functional},不过该研究还处于概念性研究,还有LiGuangying也研究孤子的操作\cite{li2020dissipative}。既然孤子可以被操作,甚至可以实现逻辑运算,未来应该可以设计基于孤子的电子设备或者通讯设备。
\subsection{材料学中的孤子}
新材料、石墨烯
\section{宏观孤子}
孤子最初是在宏观世界下发现,所以宏观世界的孤子研究可以一门值得关注的方向,毕竟可以做直接的预测与直接的感官感受。孤子在大气中的应用可以对气象学产生影响,例如台风,土星红斑。再如海洋中孤子研究\cite{stepanyants2020nonlinear},海啸进入大陆架与孤子的关系\cite{zhao2019the},孤子甚至可以拓展到天文领域,例如黑洞中孤子\cite{dymnikova2020regular},星云、土星的非线性研究\cite{ludu2019rotating}。无论是微观还是宏观,只要系统存在非线性特性,都可以尝试用孤子的理论在建模解释。

\section{新的领域}
量子图
\end{multicols}
%参考文献

\bibliographystyle{unsrt}%

\bibliography{bibfile}
\footnote{[3]为俄文文献,字符集不支持显示,所以给出了doi号码} 

\end{document}