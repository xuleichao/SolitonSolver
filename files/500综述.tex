\documentclass{article} 
\usepackage[UTF8, scheme = plain]{ctex}
\usepackage{multicol}
\usepackage{fontspec}

\begin{document}

\title{500篇孤子摘要总结}

\author{Leichao Xu and Tie Zhang}

\date{\today}



\maketitle
%摘要
\begin{abstract}
从web of knowledge中搜索关键词soliton, 导出排名500的文献及摘要,通过阅读部分文献,了解当前孤子研究的主流方向,
例如:特殊材料中孤子的产生以及其稳定性研究,光纤中孤子解的研究,利用孤子来产生逻辑门的研究,以及求孤子解的问题等,另外有一些前沿性的研究,如
超光速孤子的概念,量子图概念。此外,还有一些研究宏观世界下孤子的问题,例如高速运行的火车头部的孤子解,海洋中鱼群相关的孤子理论以及黑洞孤子概念等。
\end{abstract}
\begin{multicols}{2}
\section{Indroduction}
孤子最开始来源于水波波峰长时间保持形状传播,之后产生了非线性的研究,即孤子解属于非线性系统的解。对于微观系统,如玻色爱因斯坦系统(BEC),非线性周期系统,会产生孤子解,同时相同带隙中的孤子解与对应非线性布洛赫波有对应关系\cite{ref1}。而对于宏观系统,孤子解依然可以解释一些自然现象,例如对黑洞中孤子的研究\cite{ref2},海洋中鱼群与孤子的关系\cite{ref3},高速运行的火车头部的空气研究\cite{ref4}。无论在微观领域还是宏观领域,孤子的研究与应用非常广泛,本篇综述将从孤子的研究现状,研究应用,以及当前孤子研究的新方向与新领域做大致的阐述,一方面作为文献摘要阅读的总结,也为孤子研究的方向确定做总结,该综述的严谨性会比较差,有错误的对方或者有争议的地方,望指出或一起讨论。

%孤子研究现状
\section{孤子研究现状}
孤子研究大致可以分为两个方向,第一是理论研究与数值模拟,第二是实验研究。{todo1}
\end{multicols}
%参考文献
\begin{thebibliography}{99}  
\bibitem{ref1}Zhang, Y. P.; Wu, B. (2009): Composition Relation between Gap Solitons and Bloch Waves in Nonlinear Periodic Systems. In Physical review letters 102 (9). 
\bibitem{ref2}Dymnikova I . Regular black holes and self-gravitating solitons with DE interiors[J]. International Journal of Modern Physics A, 2020, 35(02n03):63-.
\bibitem{ref3}10.31857/S2686739720060031
\bibitem{ref4}10.31857/S2686739720060031
\end{thebibliography}
\footnote{ref3,ref4为俄文文献,字符集不支持显示,所以给出了doi号码} 
\end{document}

